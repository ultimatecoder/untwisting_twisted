\documentclass{article}
\usepackage{url}
\begin{document}

  \section{Introduction to Twisted}
    [ADD Examples of Twisted]
  \section{Event-driven Programming paradigm}
  Single threaded or sequential programming is in which requests are processed
  by a single thread. If the thread is doing some blocking task like IO then
  the program can't handle any other requests until the request is processed.

  Example
  \begin{verbatim}
    from 
  \end{verbatim}

  \section{Reactor}
    Reactor is the core of the event loop. It is responsible for waiting for
    the event and notify the program when the event is occurred. Reactor is
    dependent on platform specific libraries. Twisted will automatically choose
    default platform specific library.

    For under the hood details, In GNU/Linux platform it tries to fetch for
    epoll \cite{epoll}.  If epoll \cite{epoll} is not available, it will use
    poll \cite{poll}. All the POSIX compliant platforms it will use poll. All
    other platforms such as Windows and Macintosh it will use select reactor.

    You have to register your callbacks to the reactor and have to start it.
    Once the reactor is start, it will listen to the events and notifies to the
    program forever or until \texttt{reactor.stop} is called.

    \begin{verbatim}
      // Example reactor_server.py
      from twisted.internet import reactor

      reactor.listenTCP(8000, MyFactory())
      reactor.run()
    \end{verbatim}

  \selection{Transport}

    Transport is responsible to provide the behaviour for transferring data to
    the other end of the connection. The transport behaves differently
    according to the connection for example, TCP or UDP will have different
    implementation of transport. All the transport implementations should be
    dependent on \texttt{ITransport} interface.

    Common methods are

    \subsection{write} Should write data to the connection.

    \subsection {writeSequence} Should write list of strings to the connection.

    \subsection{loseConnection} Should close the connection after writing
    ending data.

    \subsection{getPeer} Should return remote address of the connection.

  \section{Protocol}
    Protocol will define the behaviour to process the network events in async
    manner. Twisted does have in-built definition for many protocols such as
    HTTP, Telnet, DNS etc. All the protocol classes should implement
    \texttt{IProtocol} interface. Base class is \texttt{protocol.Protocol}.

    Common methods are

    \subsection{makeConnection} Should create a connection.

    \subsection{connectionMade} Should be called when a new connection is made.

    \subsection{dataReceived} Should be called when any data is received from
    the other end of the circuit.

    \subsection{connectionLost} Should be called when the connection is
    terminated.

  \section{Protocol Factory}
    The factory will produce the instance of \texttt{Protocol} when the new
    connection is made. \texttt{Protocol} will be garbage collected when the
    collection is called. The Factory will store configuration details for
    \texttt{Protocol} instances. The factory is following
    \texttt{IProtocolFactory} interface. There client implementation of common
    factory methods is \texttt{ClientFactory}.

    Methods

    \subsection{buildProtocol} should create the instance of Protocol class.

  \begin{thebibliography}{2}
    \bibitem{epoll}
      \url{http://man7.org/linux/man-pages/man7/epoll.7.html}% chktex 8
    \bibitem{poll}
      \url{http://man7.org/linux/man-pages/man2/poll.2.html}% chktex 8
  \end{thebibliography}
\end{document}
